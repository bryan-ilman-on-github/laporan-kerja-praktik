% Definisikan judul laporan dalam bahasa Indonesia dan Inggris.
\def\judul{Eksperimen \textit{End-To-End} untuk \textit{Kafka Resource Management} pada penelitian ``Efficient topic partitioning of Apache Kafka for high-reliability real-time data streaming applications'' di lab CSL Fakultas Ilmu Komputer UI}
\def\judulInggris{End-To-End Experiments for Kafka Resource Management in the research ``Efficient topic partitioning of Apache Kafka for high-reliability real-time data streaming applications'' di lab CSL Fakultas Ilmu Komputer UI}

% Tipe laporan, misalnya: Laporan Kerja Praktik, Kampus Merdeka, Skripsi, Tugas Akhir, Thesis, atau Disertasi.
\def\type{Laporan Kerja Praktik}

% Jenjang studi penulis, bisa berisi: Diploma, Sarjana, Magister, atau Doktor.
\def\jenjang{Sarjana}

% Nama lengkap penulis laporan.
\def\penulisSatu{Bryan Raihan 'Ilman} % nama lengkap penulis pertama
\def\penulisDua{} % nama lengkap penulis kedua
\def\penulisTiga{} % nama lengkap penulis ketiga

% NPM untuk setiap penulis.
\def\npmSatu{2106704351} % NPM penulis pertama
\def\npmDua{} % NPM penulis kedua
\def\npmTiga{} % NPM penulis ketiga

% Program studi yang diambil oleh penulis.
\def\programSatu{Ilmu Komputer} % program studi penulis pertama
\def\programDua{} % program studi penulis kedua
\def\programTiga{} % program studi penulis ketiga

% Program studi dalam bahasa Inggris.
\def\studyProgramSatu{Computer Science} % 1st author's study program
\def\studyProgramDua{} % 2nd author's study program
\def\studyProgramTiga{} % 3rd author's study program

% Nama pembimbing dan penguji laporan.
\def\pembimbingSatu{Ichlasul Affan, M.Kom.}
\def\pembimbingDua{}
\def\pembimbingTiga{}

% Fakultas tempat penulis menempuh studi.
\def\fakultas{Ilmu Komputer}
\Var{\bulanTahun}{Agustus 2024}

% Gelar yang akan diperoleh setelah menyelesaikan laporan ini.
\def\gelar{Sarjana}
\def\tanggalLulus{Agustus 2024}

% Daftar judul bab pada laporan.
% Ubah judul sesuai dengan kebutuhan Anda.
\Var{\kataPengantar}{Kata Pengantar}
\Var{\babSatu}{Pendahuluan}
\Var{\babDua}{Isi}
\Var{\babTiga}{Penutup}
\Var{\babEmpat}{}
\Var{\babLima}{}
\Var{\kesimpulan}{}

% Jangan mengubah bagian di bawah ini.
\Var{\Judul}{\judul}
\Var{\Type}{\type}
\Var{\PenulisSatu}{\penulisSatu}
\Var{\PenulisDua}{\penulisDua}
\Var{\PenulisTiga}{\penulisTiga}
\Var{\Fakultas}{\fakultas}
\Var{\ProgramSatu}{\programSatu}
\Var{\ProgramDua}{\programDua}
\Var{\ProgramTiga}{\programTiga}
