\chapter{\babTiga}
\label{bab:3}

Pada bab ini dijelaskan kesimpulan dari kerja praktik yang telah dilaksanakan di Computer Systems Laboratory (CSL). Penjelasan dimulai dengan kesimpulan secara menyeluruh tentang pengalaman yang diperoleh selama kerja praktik, diikuti dengan saran untuk beberapa pihak.

\section{Kesimpulan}

Pelaksanaan kerja praktik di Computer Systems Laboratory (CSL) Fakultas Ilmu Komputer UI memberikan banyak pengalaman berharga, baik dari sisi teknis maupun non-teknis. Pelaksana kerja praktik mendapatkan wawasan mendalam tentang pengelolaan data \textit{real-time}, pengoptimalan sumber daya, serta penerapan \textit{containerization} menggunakan Docker.

Dari sisi non-teknis, pelaksana kerja praktik juga mempelajari pentingnya komunikasi efektif, baik dengan dosen peneliti maupun tim di CSL, serta dengan pihak eksternal yang terkait dengan penelitian. Salah satu pengalaman penting adalah pertukaran email dengan Profesor Claudio Cicconetti dari Università di Pisa, penulis dari penelitian yang eksperimennya sedang direplikasi.

\section{Saran}

Berdasarkan pelaksanaan kerja praktik secara keseluruhan, terdapat beberapa saran yang ingin pelaksana sampaikan kepada pelaksana kerja praktik berikutnya dan Fakultas Ilmu Komputer UI. Adapun mengenai tempat kerja praktik, pelaksana menilai bahwa pelaksanaannya sudah berjalan dengan baik dan efektif, sehingga tidak ada saran yang perlu disampaikan terkait hal tersebut.

\subsection{Pelaksana Kerja Praktik Berikutnya}

Bagi mahasiswa yang akan melaksanakan kerja praktik di CSL, disarankan untuk mempersiapkan manajemen waktu yang baik, terutama jika kerja praktik dilakukan bersamaan dengan perkuliahan aktif. Kerja praktik di CSL menawarkan fleksibilitas dalam skema \textit{hybrid}, namun tetap membutuhkan fokus yang tinggi dalam menjalankan eksperimen dan memenuhi tanggung jawab akademik secara bersamaan. Penting untuk selalu berkomunikasi dengan dosen pembimbing atau anggota tim jika mengalami kesulitan teknis agar kendala dapat segera diatasi dan waktu tidak terbuang percuma. Selain itu, mahasiswa harus siap dengan tantangan teknis terkait infrastruktur dan manajemen sumber daya, serta mampu beradaptasi dengan solusi-solusi teknologi terbaru yang mungkin berbeda dari yang dipelajari di perkuliahan.

\subsection{Fakultas Ilmu Komputer UI}

Secara umum, pelaksanaan kerja praktik di Fakultas Ilmu Komputer UI sudah berjalan dengan baik, terutama dalam hal waktu dan fleksibilitas bagi pelaksana. Namun, disarankan agar Fakultas Ilmu Komputer UI memperluas kerja sama dengan berbagai institusi dan perusahaan untuk membuka lebih banyak kesempatan kerja praktik bagi mahasiswa. Hal ini penting terutama dalam menghadapi tantangan \textit{tech winter}, di mana peluang kerja praktik di bidang teknologi lebih terbatas. Menjalin kemitraan yang lebih kuat dengan berbagai perusahaan akan memberikan mahasiswa peluang lebih besar untuk mengaplikasikan pengetahuan mereka dalam lingkungan kerja nyata.
